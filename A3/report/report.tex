\documentclass[12pt]{article}
\usepackage{graphicx}
\usepackage{paralist}
\usepackage{amsfonts}
\oddsidemargin 0mm
\evensidemargin 0mm
\textwidth 160mm
\textheight 200mm
\renewcommand\baselinestretch{1.0}
\pagestyle {plain}
\pagenumbering{arabic}
\newcounter{stepnum}
\title{Assignment 3: Part 1}
\author{COMP SCI 2ME3 and SFWR ENG 2AA4}
\begin {document}
\maketitle
\newpage


\section* {Constants Module}
\subsection*{Module}
Constants
\subsection* {Uses}
N/A
\subsection* {Syntax}
\subsubsection* {Exported Constants}
MAX\_X = 180 {\it //dimension in the x-direction of the problem area}\\
MAX\_Y = 160 {\it //dimension in the y-direction of the problem area}\\ 
TOLERANCE = 5 {\it //space allowance around obstacles}\\
VELOCITY\_LINEAR = 15 {\it //speed of the robot when driving straight}\\
VELOCITY\_ANGULAR = 30 {\it //speed of the robot when turing rad}
\subsubsection* {Exported Access Programs}
none
\subsection* {Semantics}
\subsubsection* {State Variables}
none
\subsubsection* {State Invariant}
none
\newpage


\section* {Point ADT Module}
\subsection*{Template Module}
PointT
\subsection* {Uses}
Constants
\subsection* {Syntax}
\subsubsection* {Exported Types}
PointT = ?
\subsubsection* {Exported Access Programs}
\begin{tabular}{| l | l | l | l |}
\hline
\textbf{Routine name} & \textbf{In} & \textbf{Out} & \textbf{Exceptions}\\
\hline
PointT & real, real & PointT & InvalidPointException\\
\hline
xcrd & ~ & real & ~\\
\hline
ycrd & ~ & real & ~\\
\hline
dist & PointT & real & ~\\
\hline
\end{tabular}
\subsection* {Semantics}
\subsubsection* {State Variables}
$xc$: real\\
$yc$: real
\subsubsection* {State Invariant}
none
\subsubsection* {Assumptions}
The constructor PointT is called for each abstract object before any other access routine is called for that
object.  The constructor cannot be called on an existing object.
\subsubsection* {Access Routine Semantics}
PointT($x, y$):
\begin{itemize}
\item transition: $xc, yc := x, y$
\item output: $out := \mathit{self}$
\item exception
 $exc := ((\neg(0 \leq x \leq \mbox{Contants.MAX\_X}) \vee \neg(0 \leq y \leq \mbox{Constants.MAX\_Y})) \Rightarrow
\mbox{InvalidPointException})$
\end{itemize}
\noindent xcrd():
\begin{itemize}
\item output: $out := xc$
\item exception: none
\end{itemize}
\noindent ycrd():
\begin{itemize}
\item output: $out := yc$
\item exception: none
\end{itemize}
\noindent dist($p$):
\begin{itemize}
\item output: $out := \sqrt{(\mathit{self}.xc - p.xc)^2 + (\mathit{self}.yc - p.yc)^2}$
\item exception: none
\end{itemize}
\newpage


\section* {Region Module}
\subsection* {Template Module}
RegionT
\subsection* {Uses}
PointT, Constants
\subsection* {Syntax}
\subsubsection* {Exported Types}
RegionT = ?
\subsubsection* {Exported Access Programs}
\begin{tabular}{| l | l | l | l |}
\hline
\textbf{Routine name} & \textbf{In} & \textbf{Out} & \textbf{Exceptions}\\
\hline
RegionT & PointT, real, real & RegionT & InvalidRegionException\\
\hline
pointInRegion & PointT & boolean & ~\\
\hline 
\end{tabular}
\subsection* {Semantics}
\subsubsection* {State Variables}
$\mathit{lower\_left}$: PointT {\it //coordinates of the lower left corner of the region}\\
$\mathit{width}$: real {\it //width of the rectangular region}\\
$\mathit{height}$: real {\it //height of the rectangular region}
\subsubsection* {State Invariant}
None
\subsubsection* {Assumptions}
The RegionT constructor is called for each abstract object before any other access routine is called for that
object.  The constructor can only be called once.
\subsubsection* {Access Routine Semantics}
\noindent RegionT($p, w, h$):
\begin{itemize}
\item transition: $\mathit{lower\_left}, \mathit{width}, \mathit{height} := p, w, h$
\item output: $out := \mathit{self}$
\item exception $exc := ((\neg(p.xcrd()+w \leq \mbox{Constants.MAX\_X}) \vee \neg(p.ycrd()+h \\ \indent \leq \mbox{Contants.MAX\_X}) \Rightarrow \mbox{InvalidRegionException})$\\
\end{itemize}
\noindent pointInRegion($p$):
\begin{itemize}
\item output: $\mathit{out} := \exists(i:\mathbb{N}, j:\mathbb{N}  \mid i \in [0..w] \wedge j \in [0..h] : p.dist(pointT(i,j)) \leq Constants.TOLERANCE)$\\
\item exception: none
\end{itemize}
\newpage


\section* {Generic List Module}
\subsection* {Generic Template Module}
GenericList(T)
\subsection* {Uses}
N/A
\subsection* {Syntax}
\subsubsection* {Exported Types}
GenericList(T) = ?
\subsubsection* {Exported Constants}
MAX\_SIZE = 100
\subsubsection* {Exported Access Programs}
\begin{tabular}{| l | l | l | p{5cm} |}
\hline
\textbf{Routine name} & \textbf{In} & \textbf{Out} & \textbf{Exceptions}\\
\hline
GenericList & ~ & GenericList & ~\\
\hline
add & integer, T & ~ & FullSequenceException,\newline InvalidPositionException\\
\hline
del & integer & ~ & InvalidPositionException\\
\hline
setval & integer, T & ~ & InvalidPositionException\\
\hline
getval & integer & T & InvalidPositionException\\
\hline
size & ~ & integer & ~\\
\hline
\end{tabular}
\subsection* {Semantics}
\subsubsection* {State Variables}
$s$: sequence of T
\subsubsection* {State Invariant}
$| s | \leq \mathrm{MAX\_SIZE}$
\subsubsection* {Assumptions}
The GenericList() constructor is called for each abstract object before any other access routine is called for that
object.  The constructor can only be called once.
\subsubsection* {Access Routine Semantics}
GenericList():
\begin{itemize}
\item transition: $\mathit{self}.s := < >$
\item output: $\mathit{out} := \mathit{self}$
\item exception: none
\end{itemize}
\noindent add($i, p$):
\begin{itemize}
\item transition: $s := s[0..i-1] || <p> || s[i..|s|-1]$
\item exception: $exc := (|s| = \mathrm{MAX\_SIZE} \Rightarrow  \mathrm{FullSequenceException} ~ | ~ i \notin [0..|s|] \Rightarrow
\mathrm{InvalidPositionException})$
\end{itemize}
\noindent del($i$):
\begin{itemize}
\item transition: $s := s[0..i-1] || s[i+1..|s|-1]$
\item exception:  $exc := (i \notin [0..|s|-1] \Rightarrow \mathrm{InvalidPositionException})$
\end{itemize}
\noindent setval($i, p$):
\begin{itemize}
\item transition: $s[i] := p$
\item exception: $exc := (i \notin [0..|s|-1] \Rightarrow \mathrm{InvalidPositionException})$
\end{itemize}
\noindent getval($i$):
\begin{itemize}
\item output: $out := s[i]$
\item exception: $exc := (i \notin [0..|s|-1] \Rightarrow \mathrm{InvalidPositionException})$
\end{itemize}
\noindent size():
\begin{itemize}
\item output: $out := | s |$
\item exception: none
\end{itemize}
\newpage


\section* {Path Module}
\subsection* {Template Module}
PathT is GenericList(PointT)
\section* {Obstacles Module}
\subsection* {Template Module}
Obstacles is GenericList(RegionT)
\section* {Destinations Module}
\subsection* {Template Module}
Destinations is GenericList(RegionT)
\section* {SafeZone Module}
\subsection* {Template Module}
SafeZone extends GenericList(RegionT)
\subsection*{Exported Constants}
MAX\_SIZE = 1
%\section* {Path List Module}
%\subsection* {Template Module}
%PathListT is GenericList(PathT)
\newpage


\section* {Map Module}
\subsection* {Module}
Map
\subsection* {Uses}
Obstacles, Destinations, SafeZone
\subsection* {Syntax}
\subsubsection* {Exported Access Programs}
\begin{tabular}{| l | l | l | l |}
\hline
\textbf{Routine name} & \textbf{In} & \textbf{Out} & \textbf{Exceptions}\\
\hline
init & Obstacles, Destinations, SafeZone & ~ & ~\\
\hline
get\_obstacles & ~ & Obstacles & ~\\
\hline
get\_destinations & ~ & Destinations & ~\\
\hline
get\_safeZone & ~ & SafeZone & ~\\
\hline
\end{tabular}
\subsection* {Semantics}
\subsubsection*{State Variables}
$\mathit{obstacles}:$ Obstacles\\
$\mathit{destinations}:$ Destinations\\
$\mathit{safeZone}:$ SafeZone
\subsubsection* {State Invariant}
none
\subsubsection* {Assumptions}
The access routine init() is called for the abstract object before any other access routine is called.  If the map is
changed, init() can be called again to change the map.
\subsubsection* {Access Routine Semantics}
\noindent init($o, d, sz$):
\begin{itemize}
\item transition: $\mathit{obstacles}, \mathit{destinations}, \mathit{safeZone}  := o, d, sz$
\item exception: none
\end{itemize}
\noindent get\_obstacles():
\begin{itemize}
\item output: $\mathit{out} := \mathit{obstacles}$
\item exception: none
\end{itemize}
\noindent get\_destinations():
\begin{itemize}
\item output: $\mathit{out} := \mathit{destinations}$
\item exception: none
\end{itemize}
\noindent get\_safeZone():
\begin{itemize}
\item output: $\mathit{out} := \mathit{safeZone}$
\item exception: none
\end{itemize}
\newpage



\section* {Path Calculation Module}
\subsection* {Module}
PathCalculation
\subsection* {Uses}
Constants, PointT, RegionT, PathT, Obstacles, Destinations, SafeZone, Map
\subsection* {Syntax}
\subsubsection* {Exported Access Programs}
\begin{tabular}{| l | l | l | l |}
\hline
\textbf{Routine name} & \textbf{In} & \textbf{Out} & \textbf{Exceptions}\\
\hline
is\_validSegment & PointT, PointT & boolean & ~\\
\hline
is\_validPath & PathT & boolean & ~\\
\hline
is\_shortestPath & PathT & boolean & ~\\
\hline
totalDistance & PathT & real & ~\\
\hline
totalTurns & PathT & integer & ~\\
\hline
estimatedTime & PathT & real & ~\\
\hline
%sortPathList & PathListT & PathListT & ~\\
%\hline
\end{tabular}
\subsection* {Semantics}
\subsubsection*{State Variables}
none
\subsubsection* {State Invariant}
none
\subsubsection* {Assumptions}
none
\subsubsection* {Access Routine Semantics}

\noindent  is\_validSegment ($p1, p2$):
\begin{itemize}
%\item transition: $ $
\item output: $\mathit{out} := \forall(i : \mathbb{N} | i \in [0 .. \mbox{Map.\mbox{get\_obstacles}.\mbox{size}} - 1] : Map.\mbox{getval}(i).\mbox{pointInRegion}(p1..p2)$
\item exception: none
\end{itemize}

\noindent  is\_validPath ($p_1$, $p_2$):
\begin{itemize}
%\item transition: $ $
\item output: $ $
\item exception: none
\end{itemize}

\noindent  is\_shortestPath ($p$):
\begin{itemize}
%\item transition: $ $
\item output: $\mathit{out} := \forall(i: \mathbb{N} | i \in [0 .. \mbox{Map.get\_destinations}.\mbox{size} - 1] \land Map.\mbox{get\_destinations}[i].\mbox{is\_validPath} : p.\mbox{dist}(Map.\mbox{get\_destinations}[i].\mbox{getval}) < p.\mbox{dist}(Map.\mbox{get\_destinations}[i+1].\mbox{getval})) $
\item exception: none
\end{itemize}

\noindent  totalDistance ($p$):
\begin{itemize}
%\item transition: $ $
\item output: $\mathit{out} := +(i:\mathbb{Z} | 0 \leq i \leq (p.size()-1):p.getval(i).dist(p.getval(i+1)))$
\item exception: none
\end{itemize}

\noindent  totalTurns ($p$:
\begin{itemize}
%\item transition: $ $
\item output: $\mathit{out} := +(i : \mathbb{N} | i \in [0 .. p.\mbox{size}] \land  (p[i].\mbox{ycrd} - p[i-1].\mbox{ycrd} )/ (p[i].\mbox{xcrd} - p[i-1].\mbox{xcrd}) != 0 : 1)$
\item exception: none
\end{itemize}

\noindent  estimatedTime ($p$):
\begin{itemize}
%\item transition: $ $
\item output: $\mathit{out} := +(i : \mathbb{N} |  i \in [0 .. p.\mbox{size} -1] \land  (p[i].\mbox{ycrd} - p[i-1].\mbox{ycrd} )/ (p[i].\mbox{xcrd} - p[i-1].\mbox{xcrd}) = 0 : 1 ) + p.\mbox{totalTurns}$
\item exception: none
\end{itemize}



\newpage
\section{Critique of Module Interface}

\end {document}
